\documentclass[11pt,a4paper]{article}
\usepackage[utf8]{inputenc}
%\usepackage[T1]{fontenc}
\usepackage{amsmath}
\usepackage{amsfonts}
\usepackage{amssymb}
\usepackage{mathtools}
\usepackage{graphicx, color}
\usepackage[left=1.5cm,right=1.5cm,top=2cm,bottom=2cm]{geometry}

% Tables
\usepackage{float}
\usepackage{wrapfig, booktabs}
\usepackage{multirow}

% Multiple pages long table
\usepackage{longtable}
%\usepackage{tabulary}

% New colunm type
\usepackage{array}
\newcolumntype{C}{>{\ttfamily}l}	% Columntype for LaTeX commands
\newcolumntype{T}{>{\Large}l}		% Columntype for titles
\newcolumntype{S}{>{\bfseries}c}	% Columntype for subtitles
\newcolumntype{G}{>{\Large}l}		% Columntype for Greek Alphabet

\setlength{\tabcolsep}{3pt}
\renewcommand{\arraystretch}{1.3}

%\renewcommand*{\familydefault}{\rmdefault}

\begin{document}


\begin{table}[H]
\begin{tabular}{lll}
\multicolumn{3}{l}{\textbf{Roman Numerals}} \\ \midrule
1 	& I 	& unus, una, unum 	\\ 
2 	& II	& duo, duae, duo  	\\ 
3 	& III	& tres, tria 		\\ 
4 	& IV 	& quattour			\\ 
5 	& V 	& quinque 			\\ 
6 	& VI 	& sex 				\\ 
7 	& VII 	& septem 		 	\\ 
8 	& VIII 	& octo 				\\ 
9 	& IX 	& novem 			\\ 
10 	& X 	& decem 			\\  \midrule
11 	& XI 	& undecim 			\\ 
12 	& XII 	& duodecim 			\\ 
13 	& XIII 	& tredecim 			\\ 
14 	& XIV 	& quattourdecim 	\\ 
15 	& XV 	& quindecim 		\\  
16 	& XVI 	& sedecim 			\\ 
17 	& XVII 	& septendecim, 		\\ 
18 	& XVIII & duodeviginti 		\\ 
19 	& XIX 	& undeviginti 		\\ 
20 	& XX 	& viginti 			\\  \midrule
\end{tabular} 
\quad
\begin{tabular}{lll}
\multicolumn{3}{c}{} \\ \midrule
21	& XI 	& unus et viginti	\\ 
22	& XII 	& duo et viginti 	\\ 
30	& XIII 	& triginta			\\ 
40	& XXIV 	& quandraginta 		\\
50	& L 	& quinquaginta 		\\ 
60	& LX 	& sexaginta			\\ 
70	& LXX 	& septuaginta		\\  
80	& LXXX 	& octoaginta		\\
90	& XC 	& nonaginta			\\ 
100	& C 	& centum 			\\ \midrule
150	& CL	& centum quadraginta\\ 
200	& LL 	& ducenti 			\\ 
300	& LLL 	& trecenti			\\ 
400	& CD 	& quadringenti 		\\
500	& D 	& quingenti 		\\ 
600	& DC 	& sescenti			\\ 
700	& DCC 	& septimgenti		\\  
800	& DCCC 	& octingenti		\\
900& CM 	& nongenti			\\ 
1000& M 	& mille 			\\ \midrule
\end{tabular} 
\end{table}

\begin{table}[H]
\begin{tabular}{GC}
\multicolumn{2}{c}{\textbf{Greek Alphabet}} \\ \midrule
$A \ \alpha$ 					& Alpha \\ 
$B \ \beta$ 					& Beta \\ 
$\Gamma \ \gamma$				& Gamma \\ 
$\Delta \ \delta$ 				& Delta \\ 
$E \ \epsilon \ \varepsilon $ 	& Epsilon \\ \midrule
$\digamma$ 						& Digamma \\ 
$Z \ \zeta $ 					& Zeta \\ 
$H \ \eta$ 						& Eta \\ 
$\Theta \ \theta \ \vartheta$	& Theta \\ 
$I \ \iota$ 					& Iota \\ \midrule
$K \ \kappa \ \varkappa$ 		& Kappa \\ 
$\Lambda \ \lambda$ 			& Lambda \\ 
$M \ \mu$ 						& Mu \\ 
$N \ \nu$ 						& Nu \\ 
$\Xi \ \xi$ 					& Xi \\ \midrule
$O \ o$ 						& Omicron \\ 
$\Pi \ \pi \ \varpi$ 			& Pi \\ 
$P \ \rho \ \varrho$ 			& Rho \\ 
$\Sigma \ \sigma \ \varsigma$ 	& Sigma \\ 
$T \ \tau$ 						& Tau \\ \midrule
$\Upsilon \ \upsilon$ 			& Upsilon \\ 
$\Phi \ \phi \ \varphi$			& Phi \\ 
$X \ \chi$ 						& Chi \\ 
$\Psi \ \psi$ 					& Psi \\ 
$\Omega \ \omega$				& Omega \\ \midrule
\end{tabular} 
\end{table}

\end{document}